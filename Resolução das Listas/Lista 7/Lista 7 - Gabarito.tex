\documentclass[hidelinks,11pt]{book}

%These tell TeX which packages to use.
\usepackage{array,epsfig}
\usepackage{amsmath}
\usepackage{amsfonts}
\usepackage{amssymb}
\usepackage{amsxtra}
\usepackage{amsthm}
\usepackage{mathrsfs}
\usepackage{color}
\usepackage{graphicx} % adicionar figuras
\usepackage{float}
\usepackage{scalerel,stackengine}
\usepackage{times}
\usepackage[T1]{fontenc}
\usepackage{setspace} % configurar espaçamento entre linhas
\usepackage{babel} %lingua portuguesa
\usepackage{microtype}
\usepackage{subcaption} % descrição de figuras
\usepackage[utf8]{inputenc}
\usepackage{blindtext}
\usepackage[round]{natbib}
\usepackage[a4paper, margin=2cm]{geometry} % margem
\usepackage{hyperref}

%Here I define some theorem styles and shortcut commands for symbols I use often
\theoremstyle{definition}
\newtheorem{defn}{Definition}
\newtheorem{thm}{Theorem}
\newtheorem{cor}{Corollary}
\newtheorem*{rmk}{Remark}
\newtheorem{lem}{Lemma}
\newtheorem*{joke}{Joke}
\newtheorem{ex}{Example}
\newtheorem*{soln}{Solution}
\newtheorem{prop}{Proposition}

\newcommand{\lra}{\longrightarrow}
\newcommand{\ra}{\rightarrow}
\newcommand{\surj}{\twoheadrightarrow}
\newcommand{\graph}{\mathrm{graph}}
\newcommand{\bb}[1]{\mathbb{#1}}
\newcommand{\Z}{\bb{Z}}
\newcommand{\Q}{\bb{Q}}
\newcommand{\R}{\bb{R}}
\newcommand{\C}{\bb{C}}
\newcommand{\N}{\bb{N}}
\newcommand{\M}{\mathbf{M}}
\newcommand{\m}{\mathbf{m}}
\newcommand{\MM}{\mathscr{M}}
\newcommand{\HH}{\mathscr{H}}
\newcommand{\Om}{\Omega}
\newcommand{\Ho}{\in\HH(\Om)}
\newcommand{\bd}{\partial}
\newcommand{\del}{\partial}
\newcommand{\bardel}{\overline\partial}
\newcommand{\textdf}[1]{\textbf{\textsf{#1}}\index{#1}}
\newcommand{\img}{\mathrm{img}}
\newcommand{\ip}[2]{\left\langle{#1},{#2}\right\rangle}
\newcommand{\inter}[1]{\mathrm{int}{#1}}
\newcommand{\exter}[1]{\mathrm{ext}{#1}}
\newcommand{\cl}[1]{\mathrm{cl}{#1}}
\newcommand{\ds}{\displaystyle}
\newcommand{\vol}{\mathrm{vol}}
\newcommand{\cnt}{\mathrm{ct}}
\newcommand{\osc}{\mathrm{osc}}
\newcommand{\LL}{\mathbf{L}}
\newcommand{\UU}{\mathbf{U}}
\newcommand{\support}{\mathrm{support}}
\newcommand{\AND}{\;\wedge\;}
\newcommand{\OR}{\;\vee\;}
\newcommand{\Oset}{\varnothing}
\newcommand{\st}{\ni}
\newcommand{\wh}{\widehat}

%Pagination stuff.
\setlength{\topmargin}{-.3 in}
\setlength{\oddsidemargin}{0in}
\setlength{\evensidemargin}{0in}
\setlength{\textheight}{9.in}
\setlength{\textwidth}{6.5in}
\pagestyle{empty}



\begin{document}
	
	\begin{center}
		{\Large Econometria I \hspace{0.5cm} Lista 7}\\
		Profa. Lorena Hakak\\
		Entrega: 23/11/2022
	\end{center}
	
	\vspace{0.2 cm}
	
	
	
	\section*{Exercício 3}
	
\subsection*{(i)}

O efeito marginal se torna negativo no termo quadrático da variável \textit{sales}.

\subsection*{(ii)}

Depende do nível de intervalo de confiança. a estatística \textit{t} da variável $sales^2$ é de -1.86 (0,000000007/0,000000037), a 10\% nós rejeitamos a hipótese nula (que a variável não tem significância estatística) mas a 5\% não rejeitamos.

\subsection*{(iii)}

\begin{table}[!htbp] \centering 
	\caption{} 
	\label{} 
	\begin{tabular}{@{\extracolsep{5pt}}lc} 
		\\[-1.8ex]\hline 
		\hline \\[-1.8ex] 
		& \multicolumn{1}{c}{\textit{Dependent variable:}} \\ 
		\cline{2-2} 
		\\[-1.8ex] & rdintens \\ 
		\hline \\[-1.8ex] 
		salesbil & 0.301$^{**}$ \\ 
		& (0.139) \\ 
		& \\ 
		salesbil2 & $-$0.007$^{*}$ \\ 
		& (0.004) \\ 
		& \\ 
		Constant & 2.613$^{***}$ \\ 
		& (0.429) \\ 
		& \\ 
		\hline \\[-1.8ex] 
		Observations & 32 \\ 
		R$^{2}$ & 0.148 \\ 
		Adjusted R$^{2}$ & 0.090 \\ 
		Residual Std. Error & 1.788 (df = 29) \\ 
		F Statistic & 2.527$^{*}$ (df = 2; 29) \\ 
		\hline 
		\hline \\[-1.8ex] 
		\textit{Note:}  & \multicolumn{1}{r}{$^{*}$p$<$0.1; $^{**}$p$<$0.05; $^{***}$p$<$0.01} \\ 
	\end{tabular} 
\end{table} 

\subsection*{(iv)}

Estatisticamente não há diferença entre os modelos, apenas facilita na interpretação dos coeficientes utilizar as variáveis independentes em bilhões.

	\section*{Exercício 7}

O segundo modelo, além de ter um $R^2$ maior que os outros, possui um $\bar{R}^2$ maior, mostrando que a estatística F irá aumentar em sua significância conjunta. Isso se deve por utilizar a variável \textit{totemp} em log, que se mostrou mais significante do que utilizar a variável em nível, ou acrescentar sua variável na forma quadrática, como no terceiro modelo.







	
\end{document}
