\documentclass[hidelinks,11pt]{book}

%These tell TeX which packages to use.
\usepackage{array,epsfig}
\usepackage{amsmath}
\usepackage{amsfonts}
\usepackage{amssymb}
\usepackage{amsxtra}
\usepackage{amsthm}
\usepackage{mathrsfs}
\usepackage{color}
\usepackage{graphicx} % adicionar figuras
\usepackage{float}
\usepackage{scalerel,stackengine}
\usepackage{times}
\usepackage[T1]{fontenc}
\usepackage{setspace} % configurar espaçamento entre linhas
\usepackage{babel} %lingua portuguesa
\usepackage{microtype}
\usepackage{subcaption} % descrição de figuras
\usepackage[utf8]{inputenc}
\usepackage{blindtext}
\usepackage[round]{natbib}
\usepackage[a4paper, margin=2cm]{geometry} % margem
\usepackage{hyperref}

%Here I define some theorem styles and shortcut commands for symbols I use often
\theoremstyle{definition}
\newtheorem{defn}{Definition}
\newtheorem{thm}{Theorem}
\newtheorem{cor}{Corollary}
\newtheorem*{rmk}{Remark}
\newtheorem{lem}{Lemma}
\newtheorem*{joke}{Joke}
\newtheorem{ex}{Example}
\newtheorem*{soln}{Solution}
\newtheorem{prop}{Proposition}

\newcommand{\lra}{\longrightarrow}
\newcommand{\ra}{\rightarrow}
\newcommand{\surj}{\twoheadrightarrow}
\newcommand{\graph}{\mathrm{graph}}
\newcommand{\bb}[1]{\mathbb{#1}}
\newcommand{\Z}{\bb{Z}}
\newcommand{\Q}{\bb{Q}}
\newcommand{\R}{\bb{R}}
\newcommand{\C}{\bb{C}}
\newcommand{\N}{\bb{N}}
\newcommand{\M}{\mathbf{M}}
\newcommand{\m}{\mathbf{m}}
\newcommand{\MM}{\mathscr{M}}
\newcommand{\HH}{\mathscr{H}}
\newcommand{\Om}{\Omega}
\newcommand{\Ho}{\in\HH(\Om)}
\newcommand{\bd}{\partial}
\newcommand{\del}{\partial}
\newcommand{\bardel}{\overline\partial}
\newcommand{\textdf}[1]{\textbf{\textsf{#1}}\index{#1}}
\newcommand{\img}{\mathrm{img}}
\newcommand{\ip}[2]{\left\langle{#1},{#2}\right\rangle}
\newcommand{\inter}[1]{\mathrm{int}{#1}}
\newcommand{\exter}[1]{\mathrm{ext}{#1}}
\newcommand{\cl}[1]{\mathrm{cl}{#1}}
\newcommand{\ds}{\displaystyle}
\newcommand{\vol}{\mathrm{vol}}
\newcommand{\cnt}{\mathrm{ct}}
\newcommand{\osc}{\mathrm{osc}}
\newcommand{\LL}{\mathbf{L}}
\newcommand{\UU}{\mathbf{U}}
\newcommand{\support}{\mathrm{support}}
\newcommand{\AND}{\;\wedge\;}
\newcommand{\OR}{\;\vee\;}
\newcommand{\Oset}{\varnothing}
\newcommand{\st}{\ni}
\newcommand{\wh}{\widehat}

%Pagination stuff.
\setlength{\topmargin}{-.3 in}
\setlength{\oddsidemargin}{0in}
\setlength{\evensidemargin}{0in}
\setlength{\textheight}{9.in}
\setlength{\textwidth}{6.5in}
\pagestyle{empty}



\begin{document}
	
	\begin{center}
		{\Large Econometria I \hspace{0.5cm} Lista 8}\\
		Profa. Lorena Hakak\\
		Entrega: 27/11/2022
	\end{center}
	
	\vspace{0.2 cm}
	
	
	\section*{Capítulo 6}

	
	\subsection*{Exercício 6}

Para saber se vamos incluir no modelo ou não, vamos encontrar o R-quadrado ajustado:

\begin{displaymath}
	\overline{R}^2 = 1 - \frac{(1 - R^2)(n-1)}    {n-k-1}
\end{displaymath}


\begin{displaymath}
	\overline{R}^2 = 1 - \frac{(1 - 0,232)(680-1)}    {680-8-1}
\end{displaymath}

\begin{displaymath}
	\overline{R}^2 = ,0223
\end{displaymath}
Para testar ao nível de significância de 10\%, faremos o teste F:

\begin{displaymath}
	F = \frac{\frac{(R^{2}_{ir} - R^{2}_{r})}{q}}    {\frac{1 - R^{2}_{ir}}{(n-k-1)}} = \frac{\frac{(0,232 - 0,229)}{673 - 671}}    {\frac{1 - 0,232}{(671)}} = 1,31
\end{displaymath}

Como $F>1,18$ então rejeita-se a hipótese nula. Eu incluiria essas variáveis pelo teste estatístico aplicado, apesar de alterar o R-quadrado e o R-quadrado ajustado em valores muito pequenos. Há como defender a não inclusão das variáveis pelo critério da parcimônia, em escolher modelos mais simples.

	\subsection*{Exercício 8}

\subsubsection*{(i)}

Sim, devemos inserir a variável $attend$ na regressão juntamente com $alcohol$, mas não na mesma variável. Isso porque cada uma dessas variáveis possuem impactos em direção contrárias: enquanto o aumento da frequência tende a aumentar as notas, o aumento do consumo de álcool tende a diminuir as notas. 


\subsubsection*{(ii)}

Essas variáveis são importantes de serem incluídas pois ajuda no controle do nível prévio dos alunos. Caso essas variáveis não sejam inseridas, o efeito do álcool nas notas de graduação podem estar superestimadas.


	\section*{Capítulo 7}
	
\subsection*{Exercício 2}

\subsubsection*{(i)}

Para cada cigarro a mais que a mãe fuma por dia durante a gravidez esperasse que se reduza 0,5\% do peso da criança. Caso a mãe fume pelo menos 10 cigarros por dia, esperasse uma redução do peso da criança em 5\%.

\subsubsection*{(ii)}

Espera-se que uma criança branca pese aproximadamente 4,5\% a mais que uma criança não branca. A diferença é estatisticamente significante, já que sua estatística t será de 3 (0,045/0,015), sendo significante a 1\%.


\subsubsection*{(iii)}

O que o efeito nos diz é que quanto maior os anos de escolaridade da mãe, menor é o peso da criança. Essa variável não se mostrou estatisticamente significante.

\subsubsection*{(iv)}

Não será possível calcular a estatística F por conta de que os dois modelos usam observações diferentes, já que possui um número menor do que a regressão restrita (a primeira). Para poder dar certo, teríamos que utilizar a mesma base de dados para as duas equações.


\subsection*{Exercício 3}

\subsubsection*{(i)}

Sim, a variável $hsize^2$ deve ser incluída, já que sua estatística $t$ é de 4,13, sendo estatisticamente significante a 1\%. Para calcular o tamanho ótimo, temos que observar o efeito marginal do tamanho da sala:

\begin{displaymath}
	\frac{\Delta sat}{\Delta hsize} = 19,30 - 4,38 hsize = 0
\end{displaymath}

\begin{displaymath}
	hsize \approx 4,4
\end{displaymath}


\subsubsection*{(ii)}

A diferença passa a ser de aproximadamente 45,09 pontos no $SAT$. É estatisticamente significante a 1\%, já que sua estatística $t$ foi de 10,5.

\subsubsection*{(iii)}



Sendo a regressão populacional escrita por $ sat = \beta_0 + \beta_1 hsize + \beta_2 hsize^2 + \beta_3 female + \beta_4 black + \beta_5 female . black$, então o teste de hipótese seria:

\begin{center}
	$H_0 : \beta_{3} = 0$ e $\beta_{4} = 0$\\
	$H_1 : \beta_{3} = 0$ e $\beta_{4} \not = 0$
\end{center}

Além disso, a diferença entre homens negros e não negros é de 169,81 pontos no $SAT$.


\subsubsection*{(iv)}

A diferença de notas entre mulheres negras e não negras é de 107,05 pontos no $SAT$ (62,31- 169,81). Para ver se é estatisticamente significante, eu precisaria fazer um teste $t$ restringindo $\beta_4$ e $\beta_5$ do modelo populacional.
	
\subsection*{Exercício 6}

Dado que a correlação entre a variável omitida (aptidão) e o treinamento é negativa, já que o treinamento ocorreu para aqueles com menor aptidão, e o coeficiente da variável omitida será positiva (quanto maior a aptidão, maior o salário esperado) então o viés é negativo, tendo um modelo 


\subsection*{Exercício 7}

\subsubsection*{(i)}


Sendo a regressão populacional feita em (7.29):

\begin{displaymath}
	inlf = \beta_0 + \beta_1 nwifeinc + \beta_2 educ + \beta_3 exper + \beta_4 exper^2 + \beta_5 age + \beta_6 kidslt6 + \beta_7 kidsge6
\end{displaymath}

E agora utilizando como variável dependente $outlf$, temos que:

\begin{displaymath}
	outlf = \beta_0 + \beta_1 nwifeinc + \beta_2 educ + \beta_3 exper + \beta_4 exper^2 + \beta_5 age + \beta_6 kidslt6 + \beta_7 kidsge6 + u
\end{displaymath}

Como $outlf = 1 - inlf$, então:

\begin{displaymath}
	1 - inlf = \beta_0 + \beta_1 nwifeinc + \beta_2 educ + \beta_3 exper + \beta_4 exper^2 + \beta_5 age + \beta_6 kidslt6 + \beta_7 kidsge6 + u
\end{displaymath}

\begin{displaymath}
	- inlf = -1 + \beta_0 + \beta_1 nwifeinc + \beta_2 educ + \beta_3 exper + \beta_4 exper^2 + \beta_5 age + \beta_6 kidslt6 + \beta_7 kidsge6 + u
\end{displaymath}

\begin{displaymath}
	inlf = (1 - \beta_0) - \beta_1 nwifeinc - \beta_2 educ - \beta_3 exper - \beta_4 exper^2 - \beta_5 age - \beta_6 kidslt6 - \beta_7 kidsge6 - u
\end{displaymath}

\subsubsection*{(ii)}

Não acontece nada como erro padrão das variáveis, sendo que a mudança dos sinais dos betas irão mudar o sinal apenas da estatística $t$. O mesmo ocorre para o intercepto, já que:

\begin{displaymath}
	Var(1 - \beta_0) = Var(-\beta_0) = Var(\beta_0)
\end{displaymath}

\subsubsection*{(iii)}

O R-quadrado não irá se alterar. Isso porque as mesmas variáveis estão sendo modeladas, alterando apenas a estrutura da regressão. Ao invés de ver como as variáveis influenciam a probabilidade de $inlf$ ser 1, no segundo modelo eles mostram a tendência da probabilidade de ser 0.
	
\end{document}