\documentclass[hidelinks,11pt]{book}

%These tell TeX which packages to use.
\usepackage{array,epsfig}
\usepackage{amsmath}
\usepackage{amsfonts}
\usepackage{amssymb}
\usepackage{amsxtra}
\usepackage{amsthm}
\usepackage{mathrsfs}
\usepackage{color}
\usepackage{graphicx} % adicionar figuras
\usepackage{float}
\usepackage{scalerel,stackengine}
\usepackage{times}
\usepackage[T1]{fontenc}
\usepackage{setspace} % configurar espaçamento entre linhas
\usepackage{babel} %lingua portuguesa
\usepackage{microtype}
\usepackage{subcaption} % descrição de figuras
\usepackage[utf8]{inputenc}
\usepackage{blindtext}
\usepackage[round]{natbib}
\usepackage[a4paper, margin=2cm]{geometry} % margem
\usepackage{hyperref}

%Here I define some theorem styles and shortcut commands for symbols I use often
\theoremstyle{definition}
\newtheorem{defn}{Definition}
\newtheorem{thm}{Theorem}
\newtheorem{cor}{Corollary}
\newtheorem*{rmk}{Remark}
\newtheorem{lem}{Lemma}
\newtheorem*{joke}{Joke}
\newtheorem{ex}{Example}
\newtheorem*{soln}{Solution}
\newtheorem{prop}{Proposition}

\newcommand{\lra}{\longrightarrow}
\newcommand{\ra}{\rightarrow}
\newcommand{\surj}{\twoheadrightarrow}
\newcommand{\graph}{\mathrm{graph}}
\newcommand{\bb}[1]{\mathbb{#1}}
\newcommand{\Z}{\bb{Z}}
\newcommand{\Q}{\bb{Q}}
\newcommand{\R}{\bb{R}}
\newcommand{\C}{\bb{C}}
\newcommand{\N}{\bb{N}}
\newcommand{\M}{\mathbf{M}}
\newcommand{\m}{\mathbf{m}}
\newcommand{\MM}{\mathscr{M}}
\newcommand{\HH}{\mathscr{H}}
\newcommand{\Om}{\Omega}
\newcommand{\Ho}{\in\HH(\Om)}
\newcommand{\bd}{\partial}
\newcommand{\del}{\partial}
\newcommand{\bardel}{\overline\partial}
\newcommand{\textdf}[1]{\textbf{\textsf{#1}}\index{#1}}
\newcommand{\img}{\mathrm{img}}
\newcommand{\ip}[2]{\left\langle{#1},{#2}\right\rangle}
\newcommand{\inter}[1]{\mathrm{int}{#1}}
\newcommand{\exter}[1]{\mathrm{ext}{#1}}
\newcommand{\cl}[1]{\mathrm{cl}{#1}}
\newcommand{\ds}{\displaystyle}
\newcommand{\vol}{\mathrm{vol}}
\newcommand{\cnt}{\mathrm{ct}}
\newcommand{\osc}{\mathrm{osc}}
\newcommand{\LL}{\mathbf{L}}
\newcommand{\UU}{\mathbf{U}}
\newcommand{\support}{\mathrm{support}}
\newcommand{\AND}{\;\wedge\;}
\newcommand{\OR}{\;\vee\;}
\newcommand{\Oset}{\varnothing}
\newcommand{\st}{\ni}
\newcommand{\wh}{\widehat}

%Pagination stuff.
\setlength{\topmargin}{-.3 in}
\setlength{\oddsidemargin}{0in}
\setlength{\evensidemargin}{0in}
\setlength{\textheight}{9.in}
\setlength{\textwidth}{6.5in}
\pagestyle{empty}



\begin{document}
	
	
	\begin{center}
		{\Large Econometria I \hspace{0.5cm} Lista 3}\\
		Profa. Lorena Hakak\\
		Entrega: 17/10/2022
	\end{center}
	
	\vspace{0.2 cm}
	
	
	\subsection*{1.}
	
\subsubsection{(a)}

Todos os outros fatores que influenciam a fecundidade - renda, idade, contexto familiar etc. Como é um modelo de regressão simples, é muito provável que haja correlação entre o termo de erro e a variável independente \textit{educação}.

\subsubsection{(b)}

Como os outros fatores que influenciam a fecundidade - como renda e idade - estão em $u$ e possuem correlação com \textit{educ}, a análise \textit{ceteris paribus} estará comprometida.



	\subsection*{2.}
	
\subsubsection{(Exercício 3)}

(i)

	\begin{center}
	\begin{tabular}{|c|c|c|}\hline
		\textbf{Estudante} & \textbf{GPA (Y)} & \textbf{ACT (X)} \\\hline
		1 & 2,8 & 21 \\\hline
		2 & 3,4 & 24 \\\hline
		3 & 3,0 & 26 \\\hline
		4 & 3,5 & 27 \\\hline
		5 & 3,6 & 29 \\\hline
		6 & 3,0 & 25 \\\hline
		7 & 2,7 & 25 \\\hline
		8 & 3,7 & 30 \\\hline
	\end{tabular}
\end{center}

	\begin{center}
	\begin{tabular}{|c|c|c|c|c|}\hline
		\textbf{Estudante} & \textbf{$Y_i - \overline{Y}$} & \textbf{$X_i - \overline{X}$} & \textbf{$(Y_i - \overline{Y})(X_i - \overline{X})$} & $(X_i - \overline{X})^2$ \\\hline
		1 & -0,4 & -4,9 & 2,0 & 23,8 \\\hline
		2 & 0,2 & -1,9 & -0,4 & 3,5\\\hline
		3 & -0,2 & 0,1 & 0,0 & 0,0\\\hline
		4 & 0,3 & 1,1 & 0,3 & 1,3\\\hline
		5 & 0,4 & 3,1 & 1,2 & 9,8\\\hline
		6 & -0,2 & -0,9 & 0,2 & 0,8\\\hline
		7 & -0,5 & -0,9 & 0,4 & 0,8\\\hline
		8 & 0,5 & 4,1 & 2,0 & 17,0\\\hline
	\end{tabular}
\end{center}

Sendo $ \overline{Y} = 3,2$ e $\overline{X} = 25,9$:

\begin{displaymath}
	\beta_1 = \frac{\sum_{i=1}^{8} (Y_i - \overline{Y})(X_i - \overline{X})}   {\sum_{i=1}^{8} (X_i - \overline{X})^2} 
\end{displaymath}


\begin{displaymath}
	\beta_1 \approx 0,1 
\end{displaymath}

\begin{displaymath}
	\beta_0 = \overline{Y} - \beta_1 \overline{X}
\end{displaymath}
	
\begin{displaymath}
	\beta_0 \approx 0,6
\end{displaymath}

\begin{displaymath}
	\widehat{GPA} = 0,6 + 0,1 ACT
\end{displaymath}

A relação induz que o aumento da nota do ACT está correlacionado com o aumento do GPA. O $\hat{\beta}_0$ ser 0,6 significa que mesmo que a nota do estudante no ACT seja zero, é esperado uma nota 0,6 no GPA. Além disso, caso o ACT aumentasse em 5 pontos, seu GPA aumentaria em aproximadamente 0,5.\\


(ii)

\begin{displaymath}
	\widehat{GPA} = 0,6 + 0,1 ACT
\end{displaymath}

Sendo os resíduos contidos na segunda tabela, com $(Y_i - \overline{Y})$ os resíduos de $Y_i$ e $(Z_i - \overline{Z})$ os resíduos de $X_i$. A soma de $(Y_i - \overline{Y})$ dá 0,1\footnote{Isso é arredondando a média $\overline{Y}$ para 3,2. Se utilizar o valor 3,2125 a soma dos resíduos dará 8,882E-16, um valor científico para $8,882.10^{16}$.} e  $\overline{X}$ dá 0,0. \\

(iii)\\

O valor previsto para o GPA quando o ACT = 20 é aproximadamente 2,6.\\

(iv)\\

A nota do ACT explica, nessa amostra de 8 estudantes, $57,72\%$. É possível saber isso pelo parâmetro $R^2$:

\begin{displaymath}
	R^2 = 1 - \frac{SQR}{SQT}
\end{displaymath}

\begin{displaymath}
	R^2 = 1 - \frac{\sum_{i=1}^{8} \hat{u}_{i}^{2}}  {\sum_{i=1}^{8} (Y_i - \overline{Y})^2}
\end{displaymath}


\begin{displaymath}
	R^2 = 1 - \frac{\sum_{i=1}^{8} (y_i - \hat{\beta}_0 - \hat{\beta}_1 x_i)^{2}}  {\sum_{i=1}^{8} (Y_i - \overline{Y})^2}
\end{displaymath}


\begin{displaymath}
	R^2 = 1 - \frac{0,434937}  {1,02875}
\end{displaymath}

\begin{displaymath}
	R^2 = 0,5772.
\end{displaymath}

\subsubsection{(Exercício 4)}


(i)\\

O peso de nascimento previsto quando $cigs$ é igual a zero é de 119,77. Já quando $cigs$ é igual a 20 o peso dos recém-nascidos esperado é de 109,49 onças. O que isso quer dizer é que quanto maior o número de cigarros a mãe fuma, menor o peso esperado dos recém-nascidos.\\

(ii)\\

Um modelo de regressão simples não implica uma relação causal, implicando apenas correlação entre as variáveis.\\

(iii)\\

\begin{displaymath}
	125 = 119,77 - 0,514 cigs
\end{displaymath}


\begin{displaymath}
	125 = 119,77 - 0,514 cigs
\end{displaymath}

\begin{displaymath}
	cigs = -10,1751
\end{displaymath}

Segundo o modelo de regressão, a mãe deveria fumar aproximadamente -10 cigarros. O resultado não faz sentido, já que é impossível fumar uma quantidade negativas de cigarros. Para ser possível, a relação entre peso dos recém-nascidos e cigarros fumados deveria ser positiva.\\

(iv)\\

Não. No limite, essa relação poderia não ser estatisticamente significante, o que não é o caso.

	\subsection*{3.}
	


\begin{figure}[H]
	\centering
	\includegraphics[width=0.7\linewidth]{"C:/Users/Caio Forcione/Desktop/Academic Projects/01. Economics/01. Master/Y. Monitoria Econometria 1/Resolução das Listas/Lista 3/dispersao"}
	\caption*{}
\end{figure}

	\subsection*{4.}
	
		\begin{center}
		\begin{tabular}{|c|c|}\hline
		$\beta_0$ & 0,583773\\\hline
		$\beta_1$ & 0,082744\\\hline
		$R^2$ & 0,1858\\\hline
		SQR & 120,7691\\\hline
		\end{tabular}
	\end{center}

	\subsection*{5.}

		\begin{center}
	\begin{tabular}{|c|c|}\hline
		$\beta_0$ & 0,284360\\\hline
		$\beta_1$ & 0,092029\\\hline
		$\beta_2$ & 0,004121\\\hline
		$\beta_3$ & 0,022067\\\hline
		$R^2$ & 0,316\\\hline
		SQR & 101,4556\\\hline
	\end{tabular}
\end{center}


\subsubsection{(a)}

O $\beta_1$ representa o efeito de um ano de educação na variação percentual da renda, e $\beta_2$ representa o efeito de um ano de experiência profissional. Os valores de $\beta_1$ são diferentes.


\subsubsection{(b)}

O $R^2$ aumentou e o SQR diminuiu. Como a versão longa da regressão possui mais variáveis explicativas, ela é mais tem poder explicativo, logo, um $R^2$ maior. Além disso, o resíduo da regressão será menor, causando um SQR menor por consequência.

\end{document}


